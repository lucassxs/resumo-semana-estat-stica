% NÃO altere as predefinições desse template!

\documentclass{ceel}

% ===========================
%Coloque aqui pacotes adicionais, se necessário
%%===========================

% Dados do trabalho
\title{WeatherMaringa - R Package}

% Autores: o primeiro será, necessariamente, o apresentador do trabalho
% Para  autores com mesma afiliação, deverá ser colocado o mesmo número
% entre os colchetes do \author[1]
\author[1]{\underline{Lucas Stefano Xavier de Sousa}\thanks{ ra116981@uem.br }}
\author[1]{Prof. Dr. Brian Alvarez Ribeiro de Melo}


% Adicione as instituições de cada autor e indique corretamente no campo acima
\affil[1]{DES - Universidade Estadual de Maringá}


\begin{document}
	
\inserirtitulo
			
		%Insira aqui o resumo do seu trabalho
		
		\textbf{Resumo}. Nesse espaço os autores devem inserir o texto/resumo, contendo entre 200 e 250 palavras e na estrutura: introdução; objetivos; metodologia; resultados; discussão e conclusão. 
		Conforme as normas  para apresentação de trabalhos.
		
		%Adicione as palabras-chave do seu trabalho abaixo
		\bigskip
		\textbf{Palavras-Chave:} Os autores devem apresentar um conjunto de nono mínimo 3 e máximo 5 palavras-chave (em ordem alfabética) que possam identificar os principais tópicos abordados no trabalho.
				
\end{document}
